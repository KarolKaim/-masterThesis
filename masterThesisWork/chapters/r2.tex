\chapter{Rowerowe układy napędowe}

Niniejszy rozdział zawiera informacje na temat układów napędowych stosowanych w rowerach. Przedstawiono pojęcia związane z przekładniami mechanicznymi, zasadę działania konwencjonalnego układu napędowego, przeznaczonego do zastosowań w kolarstwie, oraz najciekawsze rozwiązania oferowane na rynku komercyjnym.
%______________________________________________________________________________________________________________
\section{Przekładnie mechaniczne}
Układ napędowy to zestaw urządzeń wykorzystywany do przekazywania energii mechanicznej, w skład którego wchodzi źródło energii, układy pośredniczące w przekazywaniu energii oraz odbiornik energii \cite{maszyny1}. Mianem napędu zazwyczaj określa się urządzenia pośredniczące. Najczęściej wykorzystywanymi źródłami energii są silniki, a odbiorniki energii, których zadaniem jest realizowanie odpowiednich ruchów roboczych, przyjmują różne formy zależne od aplikacji.

Układ mechaniczny wykorzystany do przeniesienia ruchu obrotowego z elementu czynnego na element bierny nazywany jest przekładnią mechaniczną. Element czynny to element napędzający, natomiast element bierny to element napędzany. Oprócz transmisji energii, przekładnie umożliwiają również zmianę parametrów ruchu - momentu obrotowego oraz prędkości obrotowej. Przekładnie mechaniczne dzieli się na trzy grupy: cięgnowe, cierne i zębate. Przekładnie cięgnowe, które zostały opisane, z powodu zastosowania w rowerowych układach napędowych, składają się z co najmniej dwóch kół, rozsuniętych względem siebie, oraz cięgna opasającego. Ze względu na rodzaj zastosowanego cięgna, wyróżnia się przekładnie pasowe oraz łańcuchowe. Przenoszenie mocy oraz momentu obrotowego jest możliwe dzięki występującym siłą tarcia pomiędzy kołem a cięgnem(połączenia cierne), lub poprzez zazębianie się koła z cięgnem (połączenia kształtowe).
\begin{figure}[h]
    \centering
    \includegraphics[scale=0.4]{przekladnie.png}
    \caption{Różne rodzaje przekładni mechanicznych: a) przekładnia cierna, b) przekładnia cięgnowa pasowa, c) przekładnia cięgnowa łańcuchowa, d - e) przekładnie zębate. Grafika opracowana na podstawie \cite{maszyny1}.}
    \label{fig:przekladnia}
\end{figure}

Wielkością charakteryzującą przekładnie jest przełożenie. Wyróżnia się przełożenie geometryczne, kinematyczne oraz dynamiczne. Przełożenie kinematyczne to stosunek prędkości kątowej koła czynnego do prędkości kątowej koła biernego \cite{przekladnie}:
\begin{equation}
    i = \frac{\omega_1}{\omega_2}
    \label{eq:przelozenieKinematyczne}
\end{equation}
\begin{eqwhere}[2cm]
	\item[$\omega_1$] prędkość kątowa koła czynnego,
	\item[$\omega_2$] prędkość kątowa koła biernego.
\end{eqwhere}

Przełożenie przekładni jest parametrem bezwymiarowym. Ze względu na wartość przełożenia przekładni wyróżnia się:
\begin{itemize}
\item
Przekładnie przyspieszające lub tzw. multiplikatory. Przekładnie tego typu zwiększają prędkość kątową koła biernego względem prędkości kątowej koła czynnego przy jednoczesnym zmniejszeniu momentu obrotowego koła biernego względem momentu obrotowego koła czynnego. Przełożenie takiej przekładni jest liczbą z zakresu od 0 do 1.
\item
Przekładnie redukujące lub tzw. reduktory. Przekładnie tego typu działają w sposób odwrotny do przekładni przyspieszających - zmniejszają prędkość kątową koła biernego względem prędkości kątowej koła czynnego oraz zwiększają moment obrotowy koła biernego względem momentu obrotowego koła czynnego. Przełożenie przekładni redukującej jest zawsze większe od 1.
\end{itemize} 
%____________________________________________________________________________________________________________  
\section{Układ napędowy w rowerze}
W przypadku konwencjonalnego układu napędowego stosowanego w rowerach elementem napędzającym jest rowerzysta. Siła przyłożona do ramienia korby generuje moment obrotowy, który przenoszony jest, przy pomocy mechanizmu korbowego i łańcucha, na koło zębate przymocowane na stałe do piasty tylnego koła. Moment obrotowy koła napędzającego powoduje powstanie sił obwodowych, składających się na siłę napędową, która wprawia rower w ruch postępowy.

Układ napędowy roweru wykorzystanego w pracy składa się z mechanizmu korbowego z jednym kołem zębatym, łańcucha, kasety ośmiorzędowej oraz przerzutki tylnej. Liczba zębów koła zębatego zamontowanego w mechanizmie korbowym wynosi 34. Kaseta składa się z ośmiu kół zębatych, których liczba zębów należy do zakresu od 11 do 32. Przekładnia zastosowana w rowerze, niezależnie od aktualnego biegu, ma zawsze charakter multiplikatora.

Przerzutka tylna umożliwia zmianę przełożenia układu napędowego. Działa na zasadzie czworoboku przegubowego. Składa się z czterech członów połączonych przegubowo, tworzących zamknięty łańcuch kinematyczny. Pozycja wózka przerzutki, przymocowanego do jednego z członów, (Rys.\ref{fig:przerzutka}) utrzymywana jest dzięki zastosowaniu sprężyny napinającej oraz cięgna, połączonego z mechanizmem do zmiany przełożeń, zamontowanym w manetce. Dodatkowo przerzutka posiada mechanizm napinający łańcuch, tak aby zachować jego odpowiedni zwis, gwarantujący pełną funkcjonalność układu napędowego, niezależnie od aktualnej pozycji wózka przerzutki. 
\begin{figure}[h]
    \centering
    \includegraphics[scale=0.4]{przerzutka.jpg}
    \caption{Schemat ilustrujący zasadę działania konwencjonalnej tylnej przerzutki rowerowej.}
    \label{fig:przerzutka}
\end{figure}

\section{Układy napędowe oferowane na rynku komercyjnym}

Rosnące zainteresowanie branżą rowerową skutkuje wprowadzaniem na rynek coraz bardziej zaawansowanych technologicznie produktów. Dotyczy to właściwie każdego rodzaju części składających się na wyposażenie roweru. W momencie realizacji niniejszego projektu oferowanych na rynku jest co najmniej kilkanaście różnych rodzajów układów napędowych. Różnią się m.in. przeznaczeniem, ilością oferowanych przełożeń, zastosowanymi materiałami czy sposobem zmiany przełożeń.

Najbardziej ogólna klasyfikacja rowerowych układów napędowych to przeznaczenie. Można wyróżnić trzy główne grupy - kolarstwo szosowe, górskie oraz produkty przeznaczone do rowerów miejskich.

Kolarstwo szosowe stanowi główną siłę napędową rozwoju rowerowych układów napędowych. Coraz większe wymagania, pochodzące głównie z zawodowego peletonu, sprawiają, że producenci prześcigają się w oferowaniu coraz bardziej zaawansowanych rozwiązań. Produkty te powinny charakteryzować się jak najniższą masą, niezawodnością i precyzją działania. Układy napędowe przeznaczone do rowerów szosowych składają się dwóch przerzutek, kasety, mechanizmu korbowego oraz manetek, zazwyczaj zintegrowanych z klamkami hamulcowymi. Mechanizm korbowy wyposażony jest zazwyczaj w dwa blaty, natomiast kaseta zbudowana jest z kół zębatych, których liczba, zależna od grupy produktu, może wynosić od 8, dla napędów podstawowych, do 11 w przypadku produktów przeznaczonych dla profesjonalnych kolarzy. Podobnie, jak w branży motoryzacyjnej, nowe rozwiązania opracowywane są dla modeli z najwyższej grupy, które z czasem adaptowane są do tańszych odpowiedników. Dlatego Autor skupi się na przedstawieniu najciekawszych rozwiązań, stosowanych w najwyższych grupach napędów, oferowanych przez czołowych producentów. Są to Shimano, SRAM, Campagnolo, oraz FSA. Niska masa układów napędowych, która w przypadku grup mechanicznych nie przekracza 2000g \cite{roadGroupSetWeights}, jest możliwa do osiągnięcia poprzez zastosowanie bardzo lekkich materiałów. Klamkomanetki, ramiona mechanizmu korbowego, wózek tylnej przerzutki wykonane są z włókna węglowego. Stosowane są również kompozyty zbrojone włóknem węglowym CFRP( ang. \textit{Carbon Fiber Reinforced Plastics}), stopy aluminium i tytanu \cite{shimanoDuraAce}. W momencie realizacji projektu, bardzo dużym zainteresowaniem cieszą się tzw. elektroniczne układy napędowe, w których wózek przerzutki pozycjonowany jest przy użyciu serwomechanizmu. Każdy wymieniony wyżej producent posiada w swojej ofercie taki produkt. Są to odpowiednio DuraAce Di2, RED eTAP, Super Record EPS i K-Force WE. System oferowany przez Shimano i Capagnolo jest systemem przewodowym z pojedynczym pakietem zasilającym. Produkt firmy SRAM jest układem w pełni bezprzewodowym, wykorzystującym dedykowany protokół transmisji danych. Rozwiązanie pośrednie, oferowane przez FSA, polega na połączeniu przerzutek przez układ dystrybucji zasilania i jednostkę sterującą pracą serwomechanizmów oraz zastosowaniu bezprzewodowej transmisji danych ANT+ pomiędzy jednostką sterującą a klamkomanetkami \cite{kforce}.

Wymagania stawiane układom napędowym przeznaczonych do kolarstwa górskiego różnią się względem odpowiedników szosowych. Konieczna jest większa wytrzymałość oraz odporność na niekorzystne warunki atmosferyczne. Niska masa oraz aerodynamiczny kształt nie są priorytetem, jak ma to miejsce w kolarstwie szosowym. Charakter pokonywanych tras sprawia, że nie są osiągane aż tak wysokie prędkości. Biorąc to wszystko pod uwagę, producenci oferują układu na miarę potrzeb kolarzy. Shimano w swojej ofercie posiada elektroniczne grupy XTR i XT. Największy konkurent, SRAM, opracowuje innowacyjne napędy 1x11 i 1x12, które zaczynają powoli dominować nad klasycznymi układami z przednią przerzutką. Inżynierowie SRAM doszli do wniosku, że takie rozwiązanie ma dużo zalet. Pierwsza z nich to oszczędność masy. Możliwość wyeliminowania przedniej przerzutki, manetki i pancerza z linką sprawia, że SRAM XX1 Eagle jest około 20\% lżejszy od mechanicznej grupy Shimano XTR \cite{sramEagle}\cite{shimanoXtr} i waży zaledwie 465g. Kolejną zaletą jest możliwość montażu urządzeń do sterowania pracą zawieszenia lub wysuwaną sztycą, w miejscu manetki przedniej przerzutki. Jednak największą zaletą tej grupy produktów jest możliwość zastosowania rozwiązań zwiększających napięcie łańcucha, dzięki czemu nawet w trudnym terenie łańcuch nie spada z mechanizmu korbowego. Te rozwiązania to specjalna konstrukcja przedniego blatu, którego kształt oraz naprzemiennie zmieniająca się budowa zębów utrzymuje łańcuch w odpowiedniej pozycji( tzw. blaty typu Narrow-Wide), oraz mocne sprężyny wózków przerzutek tylnych, które mocno napinają łańcuch. Jedyną wadą tego typu napędów jest mniejsza liczba przełożeń - w stosunku do napędów z dwoma biegami przedniej przerzutki dokładnie dwa razy mniej. Z mniejszej liczby biegów może wynikać mniejszy zakres przełożeń. SRAM wprowadził wraz z grupą 1x12 kasetę, w której największe koło zębate posiada aż 50 zębów, co skutecznie niweluje problem mniejszego zakresu dostępnych przełożeń.

\begin{figure}[h]
    \centering
    \includegraphics[scale=0.12]{majaWloszczowskaBike.jpg}
    \caption{Dopasowanie przełożenia w wyniku zatrzymania pedałowania - tryb Sport}
    \label{fig:majkaBike}
\end{figure}

Napędy montowane w rowerach miejskich powinny charakteryzować się dużą niezawodnością, niską ceną oraz prostotą obsługi. Dodatkowo ważne jest, aby takie napędy były w miarę możliwości bezobsługowe. Sposób oraz charakter pokonywanych tras sprawia, że użytkownicy nie wymagają dużego zakresu przełożeń. Producenci sprzętu rowerowego nie rozwijają tej grupy napędów tak szybko, jak produktów przeznaczonych do zastosowania w kolarstwie górskim czy szosowym. Jedynym rozwiązaniem, na które według Autora warto zwrócić uwagę, są piasty wielobiegowe. Zmiana przełożenia jest możliwa dzięki zastosowaniu przekładni planetarnej, zamontowanej wewnątrz piasty tylego koła. Główne zalety tego rozwiązania to wysoki stopień integracji części mechanicznych ukrytych w piaście, brak konieczności przeprowadzania czynności serwisowych oraz prosta obsługa. Główni producenci rowerowych układów napędowych, a są to Shimano i SRAM, oferują kilka grup napędów tego rodzaju. Shimano posiada w swojej ofercie grupę Alfine oraz Nexus. SRAM oferuje produkty i-MOTION 3, Automatix oraz G8. 

   

