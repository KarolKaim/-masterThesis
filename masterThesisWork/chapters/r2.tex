\chapter{Zestawienie zagadnień wykorzystanych w pracy}
\label{cha:Zestawienie zagadnień wykorzystanych w pracy}

\section{Mechaniczne układy napędowe}
\subsection{Przekładnie mechaniczne}
Układ napędowy to zestaw urządzeń wykorzystywany do napędzania, w skład którego wchodzi źrodło energii, układy pośredniczące w przekazywaniu energii oraz odbiornik energi. Mianem napędu zazwyczaj określa się urządzenia pośredniczące. Najczęściej wykorzystywanymi źródłami enrgii są silniki a odbirniki energii, których zadaniem jest realizowanie odpowiednich ruchów roboczych, przyjmują różne formy zależne od aplikacji.

Układ mechaniczny wykorzystany do przeniesienia ruchu obrotowego z elementu czynnego na element bierny nazywany jest przekładnią mechaniczną. Oprócz transmisji energii, przekładnie umożliwają również zmianę parametrów ruchu - momentu obrotowego oraz prędkości obrotowej. Przekładnie mechaniczne dzieli się na trzy grupy: cięgnowe, cierne i zębate. Przekładnie cięgnowe, które zostały opisane ze wzgędu na zastosowanie w rowerowych układach napędowych, składają się z conajmniej dwóch kół zębatych, rozsuniętych względem siebie, oraz cięgna opasającego. Ze względu na rodzaj zastosowanego cięgna, wyróżnia się przekładnie pasowe oraz łańcuchowe. Przenoszenie siły z cięgna na koło zębate jest możliwe dzięki zastosowaniu odpowiednich połączeń - połączenia cierne, kształtowe lub stałe przymocowanie cięgna do koła zębatego.

\textcolor{red}{@TODO wrzucić schemat przekładni, dodać jakiś przypis doprzełkadni}
	  
\subsection{Układ napędowy w rowerze}
W przypadku konwencjonalnego układu napędowego stosowanego w rowerach źródłem energii jest rowerzysta. Siła przyłożona do ramienia korby generuje moment obrotowy, który przenoszony jest, przy pomocy mechanizmu korbowego i łańcucha, na koło zębate na stałe przymocowoane do piasty tylnego koła. Moment obrotowy koła napędzającego powoduje powstanie sił obwodowych, składających się na siłę napędową, która wprawia rower w ruch postępowy. 
\section{Urządzenia pomiarowe}
\subsection{Pomiar prędkości obrotowej}
\subsection{Pomiar przyspieszeń}
\subsection{Pomiar kąta nachylenia}

\section{Sterowanie}


\subsection{Zamknięty układ regulacji}
\subsection{Serwomechanizm}

\section{Kondycjonowanie sygnałów}


\subsection{Drganie styków}
\subsection{Filtr donlnoprzepustowy}