\chapter{Realizacja projektu}
%_______________________________________________________________________________________________________________
\section{Część sprzętowa}
\subsection{Przerzutka }
Zamontowanie serwomechanizmu, który odpowiada za pozycjonuje wózka przerzutki, można podzielić na kilka etapów. Pierwszy z nich to usunięcie sprężyny, która występuje w konwencjonalnych przerzutkach tylnych. Wiąże się to ze znisczeniem połączeń nitowych do których przymocowana jest sprężyna. Zniszoczne połączenia nitowe zastąpione zostały przez połączenia śrubowe. Wykorzystane zostały śruby z gwintem metrycznym M4 o średnicy 4 mm.

Następny etap to wykonanie elementu mocującego serwomechanizm do przerzutki. Element został wykonany zgodnie z projektem przedstawionym na rysunku technicznym:.\textcolor{red}{@TODO rysunek}. 
%_______________________________________________________________________________________________________________
\subsection{Integracja elementow elektronicznych}
%_______________________________________________________________________________________________________________

\section{Część programowa}
\subsection{Struktura programu}
Jakies grafy przedstawiajace generalna zasade dzialania programu plus opis
%_______________________________________________________________________________________________________________

\subsection{Szczegóły implementacji}





