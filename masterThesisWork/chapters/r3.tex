\chapter{Zestawienie zagadnień wykorzystanych w trakcie realizacji projektu}
\label{cha:Zestawienie zagadnien wykorzystanych w pracy}

W niniejszym rozdziale przedstawiono zagadnienia wykorzystane do realizacji układu zmiany przełożeń. Zamieszczono opis zasady działania serwomechanizmu, filtru dolnoprzepustowego RC oraz metod pomiarowych wielkości fizycznych, niezbędnych do poprawnego działania sterownika układu.
 
\section{Serwomechanizm - budowa i zasada działania}
Serwomechanizm to układ automatycznej regulacji, który służy do precyzyjnego sterowania silnikiem elektrycznym. Serwomechanizm wykorzystany w pracy pozycjonuje wózek przerzutki w zadanym położeniu.

Układy automatycznej regulacji w serwomechanizmach to zazwyczaj układy zamknięte z ujemną pętlą sprzężenia zwrotnego. Schemat takiego układu regulacji został przedstawiony na rys. \ref{fig:zamknietyUklad}. Zastosowanie ujemnego sprzężenia zwrotnego umożliwia osiągnięcie celu sterowania, pomimo występujących zakłóceń. Mogą to być zakłócenia związane z pomiarami wyjścia, zakłócenia działające na obiekt lub zakłócenia sterowania. Zasada działania układu ze sprzężeniem zwrotnym polega na porównaniu aktualnej wartości wielkości regulowanej $y$ z wartością zadaną $r$, wyznaczeniu wartości uchybu regulacji $e$ i pobudzenia do pracy regulatora, którego akcja redukuje błąd regulacji, pomimo ciągle działających zakłóceń $z$ \cite{zamknietyB}.
\begin{figure}[h]
    \centering
    \includegraphics[scale=0.4]{zamknietyUkladRegulacji.jpg}
    \caption{Schemat zamkniętego układu regulacji z ujemną pętlą sprzężenia zwrotnego. Opis sygnałów : $r$ – wartość zadana,  $e$ – uchyb regulacji, $u$ – sterowanie generowane przez regulator , $y$ – wyjście z obiektu, $z$ – zakłócenia działające na obiekt.}
    \label{fig:zamknietyUklad}
\end{figure}

Serwomechanizm zbudowany jest z kilku podstawowych elementów, takich jak silnik elektryczny, układ pomiarowy pozycji/prędkości wału silnika, regulator oraz układ mocy. W serwomechanizmach stosowane są silniki prądu stałego lub przemiennego. Serwomechanizmy oferowane na rynku mogą pracować w kilku trybach:
\begin{itemize}
\item
     Tryb regulacji położenia - układ automatycznej regulacji dąży do osiągnięcia zadanej pozycji wału silnika.
\item
    Tryb regulacji prędkości - układ automatycznej regulacji dąży do osiągnięcia zadanej prędkości obrotowej wału silnika.
\item
    Tryb regulacji momentu obrotowego - układ automatycznej regulacji dąży do osiągnięcia zadanego momentu obrotowego generowanego przez silnik.
\end{itemize}

Szczegóły dotyczące serwomechanizmu, wykorzystanego do realizacji projektu, zostały przedstawione w pkt. \ref{serwo}  
%____________________________________________________________________________________________________________  
\section{Pasywny filtr dolnoprzepustowy RC}

Filtr to układ, który przenosi na wyjście sygnały o określonej częstotliwości. Filtr dolnoprzepustowy przenosi na wyjście sygnały o niskiej częstotliwości, natomiast blokuje sygnały szybkozmienne. Ze względu na konstrukcję filtru, można je podzielić również na filtry pasywne, zbudowane tylko w oparciu o elementy RLC, i aktywne, wykorzystujące dodatkowo np. wzmacniacze operacyjne. 

Najprostszy filtr dolnoprzepustowy to pasywny filtr RC zbudowany z opornika o rezystancji $R$ i kondensatora o pojemności $C$ - rys. \ref{fig:filtrRC1}. 
\begin{figure}[h]
    \centering
    \includegraphics[scale=0.4]{filtrRC.jpg}
    \caption{Pasywny filtr dolnoprzepustowy RC - $u(t)$ sygnał wejściowy filtru, $y(t)$ sygnał wyjściowy filtru.}
    \label{fig:filtrRC1}
\end{figure}


Przyjmując napięcie na kondensatorze jako zmienną stanu $x(t)$, sterowanie układu jako sygnał $u(t)$ a obserwację jako sygnał wyjściowy filtru $y(t)$, oraz pamiętając o tym, że prąd płynący przez obwód jest równy zmianie ładunku na kondensatorze, można zapisać model matematyczny układu w postaci równań stanu:
\begin{equation}
\begin{gathered}
    \dot{x}(t) = -\frac{1}{RC}x(t) + u(t) \\
    y(t) = x(t)
     \label{eq:stanuRC}
\end{gathered}
\end{equation}

Zgodnie z definicją $4.1$ w pkt. \textit{Zagadnienie realizacji transmitancji} w pracy \cite{graba}, transmitancja operatorowa filtru RC jest opisana zależnością:
\begin{equation}
    G(s) = \frac{RCs}{RCs + 1}
    \label{eq:transRc}
\end{equation}

Z powyższego równania wynika, że obwód RC jest  układem inercyjnym I rzędu, którego stała czasowa wynosi:
\begin{equation}
   T = RC
    \label{eq:stalaCzasowaRC}
\end{equation}

Charakterystyka amplitudowa obiektu inercyjnego I rzędu wykazuje zdolność obiektu do tłumienia amplitudy sygnałów o wysokich częstotliwościach, co jest zgodne z zasadą działania filtru dolnoprzepustowego.
\begin{figure}[h]
    \centering
    \includegraphics[scale=0.23]{charakterystykaAmp.png}
    \caption{Przykładowa charakterystyka amplitudowa układu inercyjnego I rzędu \cite{charakterystykaAmp}.}
    \label{fig:charamp}
\end{figure}

\label{filtrRc}
%____________________________________________________________________________________________________________  
\section{Metody pomiarowe wielkości fizycznych}
\subsection{Pomiar prędkości kątowych koła i mechanizmu korbowego.}

Możliwość pomiaru prędkości kątowej jest kluczowa ze względu na wykonanie trybu automatycznej zmiany przełożeń. Pomiar prędkości kątowej mechanizmu korbowego umożliwi wyznaczenie aktualnej wartości kadencji.

Prędkość kątowa jest wielkością wektorową, jednak w rozważaniach dotyczących pracy brana pod uwagę jest jedynie chwilowa wartość prędkości kątowej. Ta zdefiniowana jest jako zmiana drogi kątowej w czasie:
\begin{equation}
    \omega = \frac{d\varphi}{dt}
    \label{eq:predkoscKatowa}
\end{equation}
gdzie:
\begin{eqwhere}[2cm]
	\item[$\varphi$] droga kątowa,
	\item[$t$] czas.
\end{eqwhere}
Wyznaczenie chwilowej wartości prędkości kątowej jest możliwe dzięki zastosowaniu czujnika zbliżeniowego załączanego magnetycznie, który jest przymocowany do ramy roweru, w niewielkiej odległości do tylnego koła. Zasada działania takiego czujnika jest analogiczna do działania zwykłego przycisku monostabilnego. Obwód w czujniku jest domyślnie rozwarty. Zwarcie następuje w momencie zbliżenia magnesu do czujnika. Magnes przyczepiony jest do szprychy roweru. Taki sam sposób pomiaru prędkości można znaleźć w komercyjnych licznikach rowerowych oferowanych na rynku. Autor zdecydował się na zastosowanie dwóch magnesów w celu zwiększenia dokładności pomiaru. Dysponując pomiarem czasu pomiędzy kolejnymi zwarciami czujnika magnetycznego, które, ze względu na zastosowanie dwóch magnesów, następują w wyniku obrotu koła o kąt $\pi$ , można wyznaczyć chwilową wartość prędkości kątowej: 
\begin{equation}
    \omega_{wheel} = \frac{\pi}{T_{wheel}}
\end{equation}
gdzie:
\begin{eqwhere}[2cm]
    \item[$\omega_{wheel}$] prędkość kątowa tylnego koła $[\frac{rad}{s}]$,
	\item[$T_{wheel}$] czas pomiędzy kolejnymi zwarciami czujnika magnetycznego koła $[s]$.
\end{eqwhere}
Prędkość kątowa mechanizmu korbowego została wyznaczona w sposób analogiczny. Zbliżeniowy czujnik magnetyczny, na stałe przymocowany do ramy roweru w bliskiej odległości mechanizmu korbowego, jest zwierany w każdym cyklu dzięki zastosowaniu magnesu, który jest przyklejony do ramienia mechanizmu korbowego. Ze względu na zastosowanie jednego magnesu, przyrost drogi kątowej jest dwa razy większy, niż w przypadku pomiaru prędkości kątowej koła roweru i jego wartość wynosi 2$\pi$:
\begin{equation}
    \omega_{crank} = \frac{2\pi}{T_{crank}}
\end{equation}
gdzie:
\begin{eqwhere}[2cm]
    \item[$\omega_{wheel}$] prędkość kątowa mechanizmu korbowego $[\frac{rad}{s}]$,
	\item[$T_{crank}$] czas pomiędzy kolejnymi zwarciami czujnika magnetycznego mechanizmu korbowego $[s]$.
\end{eqwhere}
\subsection{Prędkość liniowa roweru}
Wartość prędkości liniowej to stosunek przebytej drogi do czasu:
\begin{equation}
    v = \frac{s}{t}
\end{equation}
gdzie:
\begin{eqwhere}[2cm]
	\item[$s$] droga liniowa,
	\item[$v$] prędkość liniowa.
\end{eqwhere}
 Związek pomiędzy drogą liniową a drogą kątową, punktu poruszającego się po okręgu o promieniu $r$,wyraża się wzorem:

\begin{equation}
    \varphi = \frac{s}{r}
\end{equation}

W układzie pomiarowym prędkości kątowej koła roweru, który wykorzystuje dwa magnesy, wartość drogi kątowej w momencie zwarcia czujnika magnetycznego wynosi $\pi$. Dysponując pomiarem czasu $T_{wheel}$, chwilowa wartość prędkości liniowej tylnego koła, a zarazem całego roweru, może zostać wyznaczona z zależności:
 \begin{equation}
    \label{eq:zaleznoscNaPredkosc}
    v_{bike} = \frac{\pi R}{T_{wheel}}
\end{equation}
gdzie:
\begin{eqwhere}[2cm]
    \item[$v_{bike}$] prędkość roweru $[\frac{m}{s}]$,
	\item[$R$] promień koła rowerowego $[m]$.
\end{eqwhere}
%____________________________________________________________________________________________________________  
\subsection{Inercyjna jednostka pomiarowa IMU}
Inercyjna jednostka pomiarowa (ang. {\em Inertial Measurement Unit}) to układ składający się z kilku czujników pomiarowych, pozwalających wyznaczyć orientację obiektu w przestrzeni trójwymiarowej. Stosowane m.in. w systemach stabilizacji bezzałogowych statków powietrznych. W realizowanym projekcie wykorzystana zostanie możliwość wyznaczenia kąta nachylenia podłoża, po którym porusza się rower z zamontowaną jednostką pomiarową IMU.

Jednostki IMU zazwyczaj wyposażone są w trzyosiowy akcelerometr, trzyosiowy żyroskop oraz magnetometr. Wszystkie czujniki to urządzenia wykonane w miniaturowej skali oraz technologii MEMS ( ang. {\em Micro Electro-Mechanical Systems}), które integrują elementy mechaniczne oraz elektroniczne. Wykorzystywane głównie jako czujniki przetwarzające wielkości mechaniczne na wielkości elektryczne. Czujniki wykonane w technologii MEMS posiadają kilka znaczących zalet, które sprawiają, że spektrum zastosowań staje się coraz szersze. Są to m.in. niska cena, niewielkie rozmiary, niskie zużycie energii oraz prosta integracja z układami mikroprocesorowymi. 

%____________________________________________________________________________________________________________  
\subsubsection{Pomiar przyspieszeń - akcelerometr}
Akcelerometr to urządzenie pomiarowe, które umożliwia pomiar przyspieszenia dynamicznego ciała, na które działa niezerowa siła wypadkowa, oraz przyspieszenia statycznego ciała znajdującego się w ziemskim polu grawitacyjnym. Akcelerometry zazwyczaj działają na zasadzie przetworników pojemnościowych. Pomiar dokonywany jest dzięki zastosowaniu kondensatorów różnicowych, których ruchome okładki wychylane są z położenia równowagi pod wpływem działających sił bezwładności.

Akcelerometry są powszechnie wykorzystywane w różnego rodzaju aplikacjach. Układy sterujące poduszkami powietrznymi, systemy alarmowe w samochodach czy tzw. system ruszania na wzniesieniu to tylko niektóre przykłady z branży motoryzacyjnej \cite{stAkcel}.

Układy do pomiaru przyspieszenia wykonane w technologii MEMS zazwyczaj składają się z trzech akcelerometrów, które umożliwiają pomiar przyspieszenia w trzech różnych kierunkach. Akcelerometr został zamontowany w rowerze w taki sposób, aby kierunek osi \textit{x} był równoległy do podłoża (rys. \ref{fig:rownia}).
%____________________________________________________________________________________________________________  
\subsubsection{Pomiar prędkości kątowych - żyroskop}
Żyroskop to urządzenie pomiarowe dostarczające informacji na temat chwilowej prędkości obrotowej układu pomiarowego. Żyroskopy wykonane w technologii MEMS to czujniki pojemnościowe wykorzystujące efekt Coriolisa. Układ dwóch wibrujących mas, umieszczony na obrotowej tarczy, zmienia swoje położenie względem osi obrotu w zależności od chwilowej prędkości obrotowej. W wyniku zmiany położenia układu mas odkształceniu ulega tarcza, która stanowi okładkę kondensatora. W efekcie następuje różnicowa zmiana pojemności kondensatora, która przetwarzana jest na chwilową prędkość obrotową \cite{stGyro}.

Żyroskop, podobnie jak akcelerometr, to układ trzech czujników pozwalający na pomiar prędkości kątowej w trzech różnych kierunkach. Zamontowany został w taki sposób, aby kierunek osi żyroskopu \textit{x} oraz \textit{z}  był równoległy do podłoża.

%____________________________________________________________________________________________________________  
\section{Estymata kąta nachylenia podłoża}
Kąt nachylenia podłoża, po którym porusza się rower, to ostatnia z wielkości  brana pod uwagę przez algorytm automatycznej zmiany przełożeń. Wielkość ta wyznaczana jest pośrednio, poprzez zastosowanie filtru komplementarnego, wykorzystującego dane pomiarowe pochodzące z akcelerometru oraz żyroskopu.

%____________________________________________________________________________________________________________ 
\subsection{Analiza danych pomiarowych akcelerometru}
\label{pomiaryAkcel}
Przykład zamieszczony na rysunku \ref{fig:rownia} przedstawia ciało znajdujące się na równi pochyłej. Przyspieszenie ziemskie \textit{$g$} zostało rozłożone na kierunki składowe wzdłuż osi pomiarowych akcelerometru - \textit{$a_x$} oraz \textit{$a_y$}. Wektory składowe wraz z przyspieszeniem ziemskim stanowią trójkąt podobny do trójkąta, będącego przekrojem równi pochyłej, o kącie nachylenia oznaczonym jako $\alpha$. Stosunek długości przyprostokątnej leżącej naprzeciw kąta $\alpha$ do wartości drugiej przyprostokątnej, które odpowiadają wartością przyspieszeń mierzonym przez akcelerometr odpowiednio wzdłuż osi $x$ i $y$, wynosi $\tan{\alpha}$. Zastosowanie funkcji odwrotnej pozwala wyznaczyć kąt nachylenia równi:
\begin{equation}
    \alpha=\arctan{\frac{a_x}{a_y}}
    \label{eq:rowniaRadian}
\end{equation}
gdzie:
\begin{eqwhere}[2cm]
	\item[$\alpha$] kąt nachylenia podłoża
	\item[$a_x$] przyspieszenie akcelerometru mierzone wzdłuż osi $X$,
	\item[$a_y$] przyspieszenie akcelerometru mierzone wzdłuż osi $Y$.
\end{eqwhere}
Z zależności (\ref{eq:rowniaRadian}) można wyznaczyć wartość miary łukowej kąta wyrażonej w radianach. W odniesieniu do kąta nachylenia podłoża, po którym porusza się rower, zdecydowanie bardziej intuicyjną jednostką są stopnie. Przekształcenie miary kąta wyrażonej w radianach na miarę kąta wyrażoną w stopniach opisane jest poniższą zależnością:
\begin{equation}
    \alpha_{deg}=\frac{180\alpha}{\pi}
    \label{eq:rowniaStopnie}
\end{equation}
gdzie:
\begin{eqwhere}[2cm]
	\item[$\alpha_{deg}$] kąt nachylenia wyrażony w stopniach.
\end{eqwhere}
\begin{figure}[h]
    \centering
    \includegraphics[scale=0.5]{katNachylenia.jpg}
    \caption{Równia pochyła - przykład ilustrujący sposób wyznaczenia kąta nachylenia terenu.}
    \label{fig:rownia}
\end{figure}
%____________________________________________________________________________________________________________ 
\subsection{Analiza danych pomiarowych żyroskopu}
\label{gyro}
Zgodnie z zależnością (\ref{eq:predkoscKatowa}) prędkość kątowa określana jest jako zmiana drogi kątowej w czasie. Sytuacja odwrotna, czyli wyznaczenie drogi kątowej na podstawie pomiarów prędkości kątowej polega na znalezieniu funkcji odwrotnej:
\begin{equation}  
    \alpha = \int{\omega_{z}(t)dt}
    \label{eq:zyroRad}
\end{equation}
gdzie:
\begin{eqwhere}[2cm]
	\item[$\omega_{z}(t)$] prędkość kątowa wzdłuż osi $Z$ żyroskopu.
	
\end{eqwhere}

Podobnie, jak zależność \ref{eq:rowniaRadian}, równanie \ref{eq:zyroRad} pozwala wyznaczyć miarę łukową wyrażoną w radianach.

Operacja całkowania prędkości kątowej została uwzględniona w projekcie filtru komplementarnego w postaci dodatkowego członu całkującego, którego transmitancja operatorowa wynosi $\frac{1}{s}$ - wzór (\ref{eq:alpha}) w pkt. \ref{kompZasadaDzialania}.

%____________________________________________________________________________________________________________ 
\subsection{Filtr komplementarny}
\label{kompZasadaDzialania}
Zasada działania filtru komplementarnego polega na odpowiedniej fuzji danych pochodzących z różnych czujników pomiarowych. Sygnał każdego z nich zawiera istotną informację o mierzonej wielkości jak również zakłócenia, które powinny zostać wyeliminowane. Fuzja danych polega na zastosowaniu odpowiednich filtrów, dolnoprzepustowych, środkowoprzepustowych i górnoprzepustowych, które eliminują zakłócenia charakterystyczne dla danego czujnika, jednocześnie przenosząc informacje użyteczne. Warunkiem koniecznym uzyskania dokładniejszych wartości estymowanych, od wartości mierzonych, jest zastosowanie czujników pomiarowych o różnej charakterystyce częstotliwości błędów pomiarowych. Schemat przedstawiający ogólną idę filtru komplementarnego zamieszczono na rys \ref{fig:kompGeneral}.
\begin{figure}[h]
    \centering
    \includegraphics[scale=0.3]{filtrKomp.png}
    \caption{Zasada działania filtru komplementarnego. $x_1$ : $x_N$ - sygnały wejściowe, $G_1(s) : G_N(s)$ - transmitancje operatorowe poszczególnych filtrów, $\hat{x}$ - estymowana wartość wyjściowa.}
    \label{fig:kompGeneral}
\end{figure}

Zaprojektowany filtr nie powinien wprowadzać dodatkowej dynamiki - transmitancja operatorowa, określająca stosunek transformaty Laplace'a sygnału wyjściowego do transformaty Laplace'a sygnału wejściowego, powinna spełniać zależność:
\begin{equation}
    G(s)=\sum_{i=1}^{N}G_{i}(s)=1
    \label{eq:zasadaKomplementarnosci}
\end{equation}

W odniesieniu do estymacji wartości kąta nachylenia podłoża, schemat filtru komplementarnego, wykorzystującego wartości kątów wyznaczonych na podstawie danych pomiarowych pochodzących z akcelerometru oraz żyroskopu, przedstawiony został na rys \ref{fig:kompKat}. 

Pomiar składowych przyspieszenia ziemskiego umożliwia wyznaczenie kąta nachylenia podłoża (pkt. \ref{pomiaryAkcel}). Ta metoda znajduje swoje zastosowanie wtedy, gdy układ pomiarowy nie porusza się lub porusza się ruchem jednostajnym. W ruchu przyspieszonym dane pomiarowe akcelerometru uwzględniają siły powodujące ten ruch, a brak możliwości odseparowania składowych przyspieszenia ziemskiego sprawia, że kąt nachylenia obarczony jest błędem. Należy zatem zastosować filtr dolnoprzepustowy, eliminujący szybkozmienne wartości kąta nachylenia pochodzące z akcelerometru.

Żyroskop umożliwia pomiar prędkości obrotowej. Całkowanie prędkości obrotowej umożliwia wyznaczenie kąta nachylenia podłoża (pkt. \ref{gyro}). Jednak zakłócenia toru pomiarowego oraz wrażliwość na warunki pracy, a w szczególności temperaturę, powodują zjawisko tzw. dryfu żyroskopowego. Dryf spowodowany jest ciągłym narastaniem błędów w wyniku całkowania zakłóconych danych pomiarowych. Błędy powstałe w wyniku dryfu żyroskopowego są sygnałami wolnozmiennymi. Zastosowanie filtru górnoprzepustowego pozwala je wyeliminować. 

Filtr dolnoprzepustowy może zostać zrealizowany poprzez obiekt inercyjny I rzędu(pkt. \ref{filtrRc}):
\begin{equation}
    G_{low}(s)=\frac{1}{Ts+1}
    \label{eq:lowPass}
\end{equation}

Zgodnie z równaniem \ref{eq:zasadaKomplementarnosci}, filtr górnoprzepustowy powinien spełniać zależność:
\begin{equation}
    G_{high}(s)=1-G_{1}(s)=\frac{Ts}{Ts+1}
    \label{eq:highPass}
\end{equation}

Transmitancja $G_{high}(s)$ odpowiada transmitancji filtru górnoprzepustowego RC.

\begin{figure}[h]
    \centering
    \includegraphics[scale=0.34]{filtrKompKat.png}
    \caption{Schemat filtru komplementarnego estymującego wartość kąta nachylenia podłoża.}
    \label{fig:kompKat}
\end{figure}

Estymowana wartość kąta nachylenia $\hat{\alpha}$ spełnia zależność:
\begin{equation}
    \hat{\alpha} = \frac{1}{Ts+1}\alpha_{acc}+\frac{Ts}{Ts+1}\frac{1}{s}\omega = \frac{\alpha_{acc}+T\omega}{Ts + 1}
    \label{eq:alpha}
\end{equation}
gdzie:
\begin{eqwhere}[2cm]
	\item[$\hat{\alpha}$] estymowana wartość kąta nachylenia,
	\item[$\alpha_{acc}$] kąt nachylania wyznaczony na podstawie danych z akcelerometru,
	\item[$\omega$] prędkość obrotowa.
\end{eqwhere}

Ze względu na potrzebę zaimplementowania filtru w układzie mikroprocesorowym, należy znaleźć równoważny model dyskretny, który aproksymuje własności dynamiczne modelu ciągłego. Można tego dokonać stosując metodę Eulera wstecz \cite{grega}:
\begin{equation}
    s=\frac{t_0z}{z-1}
    \label{eq:eulerWstecz}
\end{equation}
gdzie:
\begin{eqwhere}[2cm]
	\item[$t_{0}$] długość kroku dyskretyzacji
\end{eqwhere}

Podstawiając zależność \ref{eq:eulerWstecz} do równania \ref{eq:alpha} oraz przyjmując podstawienie $p = \frac{T}{T + t_0}$, gdzie $T$ to stała czasowa filtru,  równanie różnicowe estymujące wartość kąta nachylania podłoża przyjmuje postać:
\begin{equation}
    \hat{\alpha_{k}} = p(\hat{\alpha}_{k-1} + \omega_{k}t_0) + (1-p)\alpha_{acc}
\end{equation}
