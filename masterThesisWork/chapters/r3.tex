\chapter{Opis podzespołów oraz narzędzi programistycznych wykorzystanych w trakcie realizacji projektu}
%_______________________________________________________________________________________________________________
\section{Wykaz podzespołów}
\subsection{Przerzutka tylna}
Autor zdecydował się na wykorzytanie tylnej przerzutki SRAM X5. Głónym atutem tej przerzutki jest jej konstrukcja, która ze względu na długie ramiona wózka przerzutki, umożliwia bezproblemowe zamontowanie serwomechanizmu. Przerzutka oferuje pełny zakres ruchu do obsługi ośmiu przełożeń. Dodatkowo jest solidną propozycją z niskiej półki cenowej.
%_______________________________________________________________________________________________________________
\subsection{Serwomechanizm}
Serwomechaznim odpowiedzialny za pozycjonowanie wózka przerzutki to Hitec Hs-8335SH. Jest to tzw. cyfrowy serwomechanizm, przeznaczony do zastosowań modelarskich. Od analogowych odpowieników odróżnia się większą precyzją w osiąganiu zadanej pozycji oraz wyższym momentem obrotowym. Autor zdecydował się na zaostosowanie gotowego serwomechanizmu z kilku powodów. Takie rozwiązanie zdecydowanie przyspieszyło prace nad projektem. W dodatku gotowy serwomechanizm charakteryzuje się kompaktowymi rozmiarami oraz odpornością na warunki atmosferyczne.  

Parametry techniczne serwomechanizmu Hitec Hs-8335SH:
\begin{itemize}
\item
Napięcie robocze [V]: 6-7.4
\item
Moment obrotowy (7.4V)[kgcm]: 24
\item
Prędkość (7.4V)[sek/$60^{\circ}$]: 0.13
\item
Metalowe tryby przekładni
\end{itemize} 
%_______________________________________________________________________________________________________________
\subsection{Mikrokontroler}
Sterownik nadzorujący pracę całego układu został wykonany w oparciu o zestaw uruchomieniowy Texas Instruments Tiva C-Series EK-TM4C123GXL. Zestaw zawiera 32-bitowy mikrokontroler TM4C123GH6PM wykorzystujący rdzeń ARM Cortex-M4F. Posiada wbubowany regulator 3.3V, port micro USB oraz wyprowadzenia niezbędnych portów I/O. Charakteryzuje się kompaktowymi rozmiarami oraz niską wagą. Mikrokontroler zastosowany w zestawie 	oferuje wysoką wydajność oraz niski pobór prądu. Kontroler przerwań sprzętowych NVIC ( z ang. {\em Nested Vectored Interrupt Controller}) umożliwia efektywne zarządzanie przewaniami w zależności od zaprogramowanych priorytetów.
\textcolor{red}{@TODO przypisik z pracyInz}

Parametry techniczne mikrokontrolera:
\begin{itemize}
\item
Zegar systemowy taktowany z częstotliwością do 80 MHz
\item
Rdzeń ARM Cortex-M4F
\item 
Obsługuje zestaw instrukcji Thumb2
\item
Pamięć flash 256kB
\item
Pamięć SRAM 32kB
\item
43 porty I/O
\item
Generator sygnału PWM
\item
Kontroler DMA
\item
Kontroler transmisji szeregowej(USB 2.0, UART, I2C, SPI, CAN)
\item
Układy czasowo-licznikowe
\end{itemize} 
%_______________________________________________________________________________________________________________
\subsection{Czujniki pomiarowe}
%_______________________________________________________________________________________________________________
\subsubsection{Akcelerometr}
Akcelerometr LSM303DLHC, którego wskazania wykorzystywane są w trybie automaycznej zmiany przełożeń, to czujnik wykonany w technologii MEMS oferowany przez firmę ST. Czujnik ten posiada również wbudowany magnetometr, jednak jego wskazania nie są wykorzystywane. Czujnik jest zamontowany w układzie Pololu AltImu10, który został wykorzystany w projekcie. Dane pomiarowe przesylane są po magistrali I2C(z ang. {\em Inter-Intergrated Circuit}). Magistrala I2C jest szerogowym, dwukierunkowym interfejsem komunikacyjnym opracowanym przez fimę Philips. Układ Pololu AltImu10 posiada elementy pasywne, gwarantujące poprawne działanie magistrali, oraz wbudowany regulator napięcia.\textcolor{red}{@TODO przypisik z pracyInz}
%_______________________________________________________________________________________________________________
\subsubsection{Zbliżeniowe czujniki magnetyczne}
Zbliżeniwe czujniki załączane magnetycznie wykorzystane są do wyznaczania prędkosci obrotowych tylnego koła oraz mechanizmu korbowego. W wyniki działania pola magnetycznego, pochodzącego z magnesu trwałego, czujnik zwiera swoje styki. 
 
Parametry techniczne:
\begin{itemize}
\item
Maksymalne napięcie pracy [V] : 50
\item
Maksymalny prąd [A]: 0.1
\item
Maksymalny zasięg [mm]: 25
\item
Styk normalnie otwarty
\end{itemize}
%_______________________________________________________________________________________________________________
\subsection{Układ zasilania}
Zródłem zasilania całego układu jest pakiet litowo-polimerowy BRAINERGY 1000mAh.
Parametry techniczne pakietu zasilającego:
\begin{itemize}
\item
Pojemność [mAh]: 500 \textcolor{red}{Ostaeczna pojemnosc pakietu ?!}
\item
Liczba ogniw: 2
\item
Napięcie robocze [V]: 7.4
\item
Wydajność prądowa [A]: 45
\end{itemize}

Dodatkowo zastosowany został regulator napięcia obniżający wartość napięcia pakietu do 5V, które jest źródłem zasilania zestawu uruchomieniowego EK-TM4C123GXL.
%_______________________________________________________________________________________________________________
\section{Narzędzia programistyczne}
\subsection{Code Composer Studio 6.0.0}
%_______________________________________________________________________________________________________________
\subsection{Biblioteka TivaWare}
%_______________________________________________________________________________________________________________
\subsection{Pozostałe oprogramowanie}




