\chapter{Wyniki pracy oraz plany rozwoju projektu}
W ramach realizacji projektu powstał w pełni funkcjonalny, elektroniczny układ do zmiany przełożeń w rowerze. Układ, zgodnie z założeniami, oferuje dwa tryby pracy - ręczny oraz automatyczny. W ramach trybu automatycznego istnieje możliwość profilowania charakterystyki zmiany przełożeń.

Układ został zainstalowany w rowerze zjazdowym. Zdjęcia kompletnego roweru wraz z głównymi elementami układu znajdują się w pkt. Dodatek A. Jazdy testowe, które obejmowały odcinki o zróżnicowanej nawierzchni oraz topografii terenu, potwierdziły skuteczność działania oraz niezawodność wykonanego układu. Jedynym problemem, który ujawnił się w trakcie testów, okazały się przyciski do zmiany przełożeń. Jeśli kciuk użytkownika, po zmianie przełożenia, pozostanie w okolicach przycisków, może dojść do mimowlonych zmian przełożeń. Wynika to z faktu, iż w trudnym terenie, rower narażony jest na duże drgania, a przyciski zwierane się w wyniku bardzo krótkiego skoku nasadki przycisku. W efekcie może dojść do przypadkowego kontaktu nasadki przycisku oraz kciuka użytkownika, co prowadzi do przypadkowych zmian przełożenia. Rozwiązaniem tego problemu może być zastosowanie filtru RC o znacznie dłuższej stałej czasowej. Jednak to rowizązanie wprowadza zdecydowane opóźnienie pomiędzy naciśnięciem przycisku a zmianą przełożenia, co według Autora pracy jest zjawiskiem niekorzystnym. Zdecydowanie lepszym pomysłem jest opracowanie systemu podobnego do konwencjonalnej manetki rowerowej, w którym przyciski aktywowane są poprzez mechanizm dźwigni o odpowiednio długim skoku. Jest to jeden z elementów projetku, który będzie rozwijany w przyszłości i polega na opracowaniu odpowiedniego układu mechanicznego.

Opracowany prototyp zostasł zbudowany w oparciu o zestaw uruchomieniowy, zewnętrzny regulator napięcia oraz dedykowanych filtrów RC, wykonanych na płytce uniwersalnej ze standardowym rastrem 2.54mm. Takie rozwiązanie zdecydowanie ułatwiło prace nad projektem, jednocześnie sprawijąc, że rozmiar sterownika układu jest zdecydowanie za duży. Dlatego plany rozwoju projektu obejmują również opracowanie dedykowanego obwodu elektronicznego. Dedykowany obwód pozwoli w znaczący sposób zminiejszyć rozmiary sterownika poprzez usunięcie nieużywanych wyprowadzeń portów I/O, interfejsu ICDI oraz połączeń przewodowych. 

