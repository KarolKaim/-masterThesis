\chapter{Eksperymenty}

W niniejszym rozdziale przedstawiony został krótki opis środowiska testowego oraz przebieg testów układu do zmiany przełożeń.

\section{Środowisko testowe}
Wszystkie eksperymenty przeprowadzone zostały w sposób stacjonarny. Akwizcyja danych sterownika układu jest w takim przypadku zdecydowanie mniej problematyczna. Zamontowanie roweru na dedykowanym stojaku umożliwia swobodną symulację dowolnych scenariuszy testowych.

Na potrzeby testów, wykonana została aplikacja przedstawiająca, w formie przebiegów czasowych, kluczowe parametry układu. Wykorzystano język C++ oraz zestaw bibliotek Qt. Komunikacja pomiędzy sterownikiem a aplikacją nawiązana została za pomocą intefrejsu szeregowego UART(ang. \textit{Universal Asynchronous Receiver and Transmitter}). Sterownik pracujący w trybie testowania, wysyła zastaw danych do komputera PC z częstotliwością 20Hz. Zestaw danych, składa się z chwilowych wartości:
\begin{itemize}
 \item
 kąta nachylenia podłoża,
 \item
 kadencji,
 \item
 prędkości liniowej roweru,
 \item
 wybranego biegu,
 \item
 stanu przycisku redukującego przełożenie,
 \item
 stanu przycisku zwiększającego przełożenie,
 \item
 trybu pracy sterownia.
 \end{itemize}
 
 Slot czasowy dla danych jest stały i wynosi 16 sekund. We wszystkich trybach automatycznych przedstawione są również przyjęte zakresy kadencji, a w trybie sportowym dodatkowo zakres kąta nachylenia podłoża.  
 
\section{Eksperymenty}