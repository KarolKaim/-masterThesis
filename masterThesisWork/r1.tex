\chapter{Wstęp}
\label{cha:Wstęp}

\section{Wprowadzenie}
\label{sec:Wprowadzenie}
Nowoczesnie technologie znajdują coraz większe zastosowanie w codziennym życiu. Szybki rozwój techniki, miniaturyzacja oraz niższe koszty produkcji sprawiają, że coraz więcej urządzeń codziennego użytku wyposażonych jest w układy elektroniczne, które skutczenie zwiększają ich możliwości. 

Obserwując rynek rowerowy można zauważyć, iż coraz większy nacisk kładziony jest na rozwój oraz zastosowanie wszelkiego rodzaju podzespołów rowerowych, wykorzystujących układy elektroniczne. Doskonałym przykładem takiego urządzenia, a właściwie zespołu urządzeń stanowiących integralną całość produktu, są elektroniczne układy do zmiany przełożeń w rowerze.

Pierwsze próby wykonania takich układów miały miejsce w latach 90 ubiegłego wieku. Jednak ze względu na wygórowaną cenę, wyższą zawodność niż odpowiedniki mechaniczne, oraz wysoką masę, która zwłaszcza w kolarstwie szosowym jest nie do zaakceptowania, nie odniosły komercyjnego sukcesu. Dopiero w roku 2009 japońska firma Shimano wprowadziła na rynek w pełni funkcjonalny układ DuraAce2(\cite{shimanoHistory}). Grupa ugruntowała swoją pozycję na rynku m.in. poprzez to, że rowery wyposażone w napęd tej grupy wygrały już kilkukrotnie wielkie wyścigi kolarskie, np. TourDeFrance.

\section{Cele pracy}
\label{sec:celePracy}

\subsection{Jakiś tytuł}
\subsubsection{Jakiś tytuł w subsubsection}
\subsection{Jakiś tytuł 2}